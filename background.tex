
\section*{Background}

\subsection*{Verification \& Validation (software v. autonomy)}
Verification Def’n: Requirements evaluation during development

Current software engineering tools for verification
Model checking (Alloy, TLA+, VCC)
Successful extensions to Internet scale (BGP, PAXOS)
Autonomy has borrowed/expanded these concepts for the verification of autonomy
Model checking (SPIN, NuSVM)
Probabilistic model checkers (PRISM)

State of the art tools are being used to verify software and autonomy, even probabilistically 

Validation Def’n: Requirements evaluation after integration

Current validation is empirical (test and eval)
Errors are a clear function of complexity*: 
- table of software errors

0.1-1.0 errors/kLOC crit
F35A:
Expect ~1800 code errors
Over half cost is software

Current validation methods are not sustainable as systems increase in complexity





\subsection*{Correct by construction controllers}

- summary here of Hadas’ stuff, others
- flow chart and example



Can automatically develop  controllers (from models, high level specifications) that are probabilistically correct




\subsection*{Probabilistic modeling and humans}

Many types of models
Motor skills
     .
     .
     .
Bayesian Networks
Gaussian Processes
     .
     .
     .
ACT-R

Increasing in capabiolities complexity

Give GP example from Ergonomics


There is a growing maturity in probabilistic models of human decision making


\subsection*{Collaborative autonomy}

Collaboration over computation, communication, navigation, and sensing

Multi-vehicle collaboration 
Decentralized task selection / allocation (Market protocols - Wellman et al, Optimization - How et al) 
Cooperative path planning (Tsourdos et al)
Scalability through swarm-based techniques (e.g., consensus, potential field)  (McLain et al)

Collaborative human-autonomy systems
Human intent prediction (e.g. partially-observable Markov Decision Process) (Karami et al)
Adaptive tasking (Parasuram et al)
Use metrics (e.g., confidence) to decide when to ask for help (Fong et al) 
Apply perspective-taking to project companion awareness state (Trafton et al)


Capable strategies for collaborative autonomy systems exist and can be 
leveraged for co-design of human-autonomy systems


\subsection*{Model-based system engineering}
Transition from functional decomposition..
Does not scale well
Difficult to verify


…to Model-based System Engineering 
State-based models of each actor are generated.
Each model is re-used over all phases of system engineering
Improves scalability and consistency
State analysis offers a formal method to verify behaviors
Successfully applied to single-actor systems (e.g., spacecraft missions)

