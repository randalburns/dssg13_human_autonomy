
\section*{Motivation and Key Research Questions}

The evolution of autonomy has revealed that the humans need to be inextricably involved in all dimensions of autonomous systems, not 
just managing those systems, but seamlessly integrating human capabilities, human information gathering, and human decision making.
\begin{center}
\parbox[c]{6in}{
{\em ``$\ldots$ the true value of unmanned systems is not to provide a direct human replacement, but rather to extend and complement human capability.''} \\
\hspace*{20pt} DSB Task Force Report, The Role of Autonomy in DoD Systems, 2012.
}
\end{center}
The DSB asserts that the benefits of Autonomy lie in unlimited persistence, reducing risk to humans, and reducing cognitive load.
Autonomy should in no way eliminate or reduces human capabilities.  This lead to shift in thinking in which autonomy is a capability to
be integrated into a complex human-machine system,  rather than a functional replacement for a human.

With this conceptual shift, the question arises: How can autonomy integrate into human-machine systems so that they maximize the 
performance the system as a whole?  Insight into system performance requires measures and models of human performance 
both standalone and in the presence of autonomy and the capability to extend performance to systems that are multi-scale
networks of humans and autonomy.

Beyond performance, it is essential to have
guarantees that autonomous systems operate correctly within bounds set by human oversight, not system specifications.
\begin{center}
\parbox[c]{6in}{
{\em ``Autonomous and semi-autonomous weapon systems shall be designed to allow 
commanders and operators to exercise appropriate levels of human judgment over the use of force.''}
\hspace*{20pt} A. B. Carter, DoD Directive 3000.09, November 21, 2012
}
\end{center}
This becomes increasingly difficult as autonomy grows in complexity, scope, and scale.  Advances are needed in 
the design of human-autonomy interaction, to ensure that autonomy reflects operator and command intent.  
Advances are also need in proving the correctness of autonomous systems.  This must include the correctness of
humans, as their capabilities and tasks are networked into autonomy and will often serve as inputs to 
autonomous capabilities.

%Ensure that autonomy doesn’t compromise our humanity
%\begin{center}
%\parbox[c]{6in}{
%{\em ``To comply with international humanitarian law, fully autonomous weapons would need human qualities that they inherently lack.''} \\
%\hspace*{20pt} Human Rights Watch, “Losing Humanity”, November 19, 2012
%%}
%\end{center}

\subsection*{Definition of Autonomy}
(DoD Autonomy Priority Steering Council, Nov 2012)
Definition of Autonomy: “capability and freedom to self-direct to achieve mission objectives”

“Military Power in the 21st Century will be defined by our ability to adapt – this is THE hallmark of autonomy”

Key Technical Challenge Areas/Gaps
Human-Autonomy Interaction and Collaboration
Scalability
Machine Intelligence
Verification and Validation

\subsection*{Key Research Questions:}

How do we model, verify, and utilize humans as collaborative decision makers and actors in correct-by-construct autonomous systems?

How do we create multi-scale, model-based autonomy (controllers, communication, and computation) that scales to the battlespace?

How do we co-design complex mission systems to optimally and acceptably incorporate human and autonomy decision-makers and actors?


