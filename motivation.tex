
\section*{Motivation and Key Research Questions}

Why Humans and Autonomy?
Humans and Autonomy must be collaborative
“…the true value of unmanned systems is not to provide a direct human replacement, but rather to extend and complement human capability” (DSB Task Force Report, 2012)

Autonomy must reflect human ethics and decision making
(autonomous systems must) “…allow commanders and operators to exercise appropriate levels of human judgment…” (A. B. Carter, DoD Directive 3000.09, November 21, 2012)

Ensure that autonomy doesn’t compromise our humanity
“To comply with international humanitarian law, fully autonomous weapons would need human qualities that they inherently lack.” (Human Rights Watch, “Losing Humanity”, November 19, 2012)

\subsection*{Definition of Autonomy}
(DoD Autonomy Priority Steering Council, Nov 2012)
Definition of Autonomy: “capability and freedom to self-direct to achieve mission objectives”

“Military Power in the 21st Century will be defined by our ability to adapt – this is THE hallmark of autonomy”

Key Technical Challenge Areas/Gaps
Human-Autonomy Interaction and Collaboration
Scalability
Machine Intelligence
Verification and Validation

\subsection*{Key Research Questions:}

How do we model, verify, and utilize humans as collaborative decision makers and actors in correct-by-construct autonomous systems?

How do we create multi-scale, model-based autonomy (controllers, communication, and computation) that scales to the battlespace?

How do we co-design complex mission systems to optimally and acceptably incorporate human and autonomy decision-makers and actors?


